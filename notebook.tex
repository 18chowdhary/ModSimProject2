
% Default to the notebook output style

    


% Inherit from the specified cell style.




    
\documentclass[11pt]{article}

    
    
    \usepackage[T1]{fontenc}
    % Nicer default font (+ math font) than Computer Modern for most use cases
    \usepackage{mathpazo}

    % Basic figure setup, for now with no caption control since it's done
    % automatically by Pandoc (which extracts ![](path) syntax from Markdown).
    \usepackage{graphicx}
    % We will generate all images so they have a width \maxwidth. This means
    % that they will get their normal width if they fit onto the page, but
    % are scaled down if they would overflow the margins.
    \makeatletter
    \def\maxwidth{\ifdim\Gin@nat@width>\linewidth\linewidth
    \else\Gin@nat@width\fi}
    \makeatother
    \let\Oldincludegraphics\includegraphics
    % Set max figure width to be 80% of text width, for now hardcoded.
    \renewcommand{\includegraphics}[1]{\Oldincludegraphics[width=.8\maxwidth]{#1}}
    % Ensure that by default, figures have no caption (until we provide a
    % proper Figure object with a Caption API and a way to capture that
    % in the conversion process - todo).
    \usepackage{caption}
    \DeclareCaptionLabelFormat{nolabel}{}
    \captionsetup{labelformat=nolabel}

    \usepackage{adjustbox} % Used to constrain images to a maximum size 
    \usepackage{xcolor} % Allow colors to be defined
    \usepackage{enumerate} % Needed for markdown enumerations to work
    \usepackage{geometry} % Used to adjust the document margins
    \usepackage{amsmath} % Equations
    \usepackage{amssymb} % Equations
    \usepackage{textcomp} % defines textquotesingle
    % Hack from http://tex.stackexchange.com/a/47451/13684:
    \AtBeginDocument{%
        \def\PYZsq{\textquotesingle}% Upright quotes in Pygmentized code
    }
    \usepackage{upquote} % Upright quotes for verbatim code
    \usepackage{eurosym} % defines \euro
    \usepackage[mathletters]{ucs} % Extended unicode (utf-8) support
    \usepackage[utf8x]{inputenc} % Allow utf-8 characters in the tex document
    \usepackage{fancyvrb} % verbatim replacement that allows latex
    \usepackage{grffile} % extends the file name processing of package graphics 
                         % to support a larger range 
    % The hyperref package gives us a pdf with properly built
    % internal navigation ('pdf bookmarks' for the table of contents,
    % internal cross-reference links, web links for URLs, etc.)
    \usepackage{hyperref}
    \usepackage{longtable} % longtable support required by pandoc >1.10
    \usepackage{booktabs}  % table support for pandoc > 1.12.2
    \usepackage[inline]{enumitem} % IRkernel/repr support (it uses the enumerate* environment)
    \usepackage[normalem]{ulem} % ulem is needed to support strikethroughs (\sout)
                                % normalem makes italics be italics, not underlines
    

    
    
    % Colors for the hyperref package
    \definecolor{urlcolor}{rgb}{0,.145,.698}
    \definecolor{linkcolor}{rgb}{.71,0.21,0.01}
    \definecolor{citecolor}{rgb}{.12,.54,.11}

    % ANSI colors
    \definecolor{ansi-black}{HTML}{3E424D}
    \definecolor{ansi-black-intense}{HTML}{282C36}
    \definecolor{ansi-red}{HTML}{E75C58}
    \definecolor{ansi-red-intense}{HTML}{B22B31}
    \definecolor{ansi-green}{HTML}{00A250}
    \definecolor{ansi-green-intense}{HTML}{007427}
    \definecolor{ansi-yellow}{HTML}{DDB62B}
    \definecolor{ansi-yellow-intense}{HTML}{B27D12}
    \definecolor{ansi-blue}{HTML}{208FFB}
    \definecolor{ansi-blue-intense}{HTML}{0065CA}
    \definecolor{ansi-magenta}{HTML}{D160C4}
    \definecolor{ansi-magenta-intense}{HTML}{A03196}
    \definecolor{ansi-cyan}{HTML}{60C6C8}
    \definecolor{ansi-cyan-intense}{HTML}{258F8F}
    \definecolor{ansi-white}{HTML}{C5C1B4}
    \definecolor{ansi-white-intense}{HTML}{A1A6B2}

    % commands and environments needed by pandoc snippets
    % extracted from the output of `pandoc -s`
    \providecommand{\tightlist}{%
      \setlength{\itemsep}{0pt}\setlength{\parskip}{0pt}}
    \DefineVerbatimEnvironment{Highlighting}{Verbatim}{commandchars=\\\{\}}
    % Add ',fontsize=\small' for more characters per line
    \newenvironment{Shaded}{}{}
    \newcommand{\KeywordTok}[1]{\textcolor[rgb]{0.00,0.44,0.13}{\textbf{{#1}}}}
    \newcommand{\DataTypeTok}[1]{\textcolor[rgb]{0.56,0.13,0.00}{{#1}}}
    \newcommand{\DecValTok}[1]{\textcolor[rgb]{0.25,0.63,0.44}{{#1}}}
    \newcommand{\BaseNTok}[1]{\textcolor[rgb]{0.25,0.63,0.44}{{#1}}}
    \newcommand{\FloatTok}[1]{\textcolor[rgb]{0.25,0.63,0.44}{{#1}}}
    \newcommand{\CharTok}[1]{\textcolor[rgb]{0.25,0.44,0.63}{{#1}}}
    \newcommand{\StringTok}[1]{\textcolor[rgb]{0.25,0.44,0.63}{{#1}}}
    \newcommand{\CommentTok}[1]{\textcolor[rgb]{0.38,0.63,0.69}{\textit{{#1}}}}
    \newcommand{\OtherTok}[1]{\textcolor[rgb]{0.00,0.44,0.13}{{#1}}}
    \newcommand{\AlertTok}[1]{\textcolor[rgb]{1.00,0.00,0.00}{\textbf{{#1}}}}
    \newcommand{\FunctionTok}[1]{\textcolor[rgb]{0.02,0.16,0.49}{{#1}}}
    \newcommand{\RegionMarkerTok}[1]{{#1}}
    \newcommand{\ErrorTok}[1]{\textcolor[rgb]{1.00,0.00,0.00}{\textbf{{#1}}}}
    \newcommand{\NormalTok}[1]{{#1}}
    
    % Additional commands for more recent versions of Pandoc
    \newcommand{\ConstantTok}[1]{\textcolor[rgb]{0.53,0.00,0.00}{{#1}}}
    \newcommand{\SpecialCharTok}[1]{\textcolor[rgb]{0.25,0.44,0.63}{{#1}}}
    \newcommand{\VerbatimStringTok}[1]{\textcolor[rgb]{0.25,0.44,0.63}{{#1}}}
    \newcommand{\SpecialStringTok}[1]{\textcolor[rgb]{0.73,0.40,0.53}{{#1}}}
    \newcommand{\ImportTok}[1]{{#1}}
    \newcommand{\DocumentationTok}[1]{\textcolor[rgb]{0.73,0.13,0.13}{\textit{{#1}}}}
    \newcommand{\AnnotationTok}[1]{\textcolor[rgb]{0.38,0.63,0.69}{\textbf{\textit{{#1}}}}}
    \newcommand{\CommentVarTok}[1]{\textcolor[rgb]{0.38,0.63,0.69}{\textbf{\textit{{#1}}}}}
    \newcommand{\VariableTok}[1]{\textcolor[rgb]{0.10,0.09,0.49}{{#1}}}
    \newcommand{\ControlFlowTok}[1]{\textcolor[rgb]{0.00,0.44,0.13}{\textbf{{#1}}}}
    \newcommand{\OperatorTok}[1]{\textcolor[rgb]{0.40,0.40,0.40}{{#1}}}
    \newcommand{\BuiltInTok}[1]{{#1}}
    \newcommand{\ExtensionTok}[1]{{#1}}
    \newcommand{\PreprocessorTok}[1]{\textcolor[rgb]{0.74,0.48,0.00}{{#1}}}
    \newcommand{\AttributeTok}[1]{\textcolor[rgb]{0.49,0.56,0.16}{{#1}}}
    \newcommand{\InformationTok}[1]{\textcolor[rgb]{0.38,0.63,0.69}{\textbf{\textit{{#1}}}}}
    \newcommand{\WarningTok}[1]{\textcolor[rgb]{0.38,0.63,0.69}{\textbf{\textit{{#1}}}}}
    
    
    % Define a nice break command that doesn't care if a line doesn't already
    % exist.
    \def\br{\hspace*{\fill} \\* }
    % Math Jax compatability definitions
    \def\gt{>}
    \def\lt{<}
    % Document parameters
    \title{Project\_2}
    
    
    

    % Pygments definitions
    
\makeatletter
\def\PY@reset{\let\PY@it=\relax \let\PY@bf=\relax%
    \let\PY@ul=\relax \let\PY@tc=\relax%
    \let\PY@bc=\relax \let\PY@ff=\relax}
\def\PY@tok#1{\csname PY@tok@#1\endcsname}
\def\PY@toks#1+{\ifx\relax#1\empty\else%
    \PY@tok{#1}\expandafter\PY@toks\fi}
\def\PY@do#1{\PY@bc{\PY@tc{\PY@ul{%
    \PY@it{\PY@bf{\PY@ff{#1}}}}}}}
\def\PY#1#2{\PY@reset\PY@toks#1+\relax+\PY@do{#2}}

\expandafter\def\csname PY@tok@w\endcsname{\def\PY@tc##1{\textcolor[rgb]{0.73,0.73,0.73}{##1}}}
\expandafter\def\csname PY@tok@c\endcsname{\let\PY@it=\textit\def\PY@tc##1{\textcolor[rgb]{0.25,0.50,0.50}{##1}}}
\expandafter\def\csname PY@tok@cp\endcsname{\def\PY@tc##1{\textcolor[rgb]{0.74,0.48,0.00}{##1}}}
\expandafter\def\csname PY@tok@k\endcsname{\let\PY@bf=\textbf\def\PY@tc##1{\textcolor[rgb]{0.00,0.50,0.00}{##1}}}
\expandafter\def\csname PY@tok@kp\endcsname{\def\PY@tc##1{\textcolor[rgb]{0.00,0.50,0.00}{##1}}}
\expandafter\def\csname PY@tok@kt\endcsname{\def\PY@tc##1{\textcolor[rgb]{0.69,0.00,0.25}{##1}}}
\expandafter\def\csname PY@tok@o\endcsname{\def\PY@tc##1{\textcolor[rgb]{0.40,0.40,0.40}{##1}}}
\expandafter\def\csname PY@tok@ow\endcsname{\let\PY@bf=\textbf\def\PY@tc##1{\textcolor[rgb]{0.67,0.13,1.00}{##1}}}
\expandafter\def\csname PY@tok@nb\endcsname{\def\PY@tc##1{\textcolor[rgb]{0.00,0.50,0.00}{##1}}}
\expandafter\def\csname PY@tok@nf\endcsname{\def\PY@tc##1{\textcolor[rgb]{0.00,0.00,1.00}{##1}}}
\expandafter\def\csname PY@tok@nc\endcsname{\let\PY@bf=\textbf\def\PY@tc##1{\textcolor[rgb]{0.00,0.00,1.00}{##1}}}
\expandafter\def\csname PY@tok@nn\endcsname{\let\PY@bf=\textbf\def\PY@tc##1{\textcolor[rgb]{0.00,0.00,1.00}{##1}}}
\expandafter\def\csname PY@tok@ne\endcsname{\let\PY@bf=\textbf\def\PY@tc##1{\textcolor[rgb]{0.82,0.25,0.23}{##1}}}
\expandafter\def\csname PY@tok@nv\endcsname{\def\PY@tc##1{\textcolor[rgb]{0.10,0.09,0.49}{##1}}}
\expandafter\def\csname PY@tok@no\endcsname{\def\PY@tc##1{\textcolor[rgb]{0.53,0.00,0.00}{##1}}}
\expandafter\def\csname PY@tok@nl\endcsname{\def\PY@tc##1{\textcolor[rgb]{0.63,0.63,0.00}{##1}}}
\expandafter\def\csname PY@tok@ni\endcsname{\let\PY@bf=\textbf\def\PY@tc##1{\textcolor[rgb]{0.60,0.60,0.60}{##1}}}
\expandafter\def\csname PY@tok@na\endcsname{\def\PY@tc##1{\textcolor[rgb]{0.49,0.56,0.16}{##1}}}
\expandafter\def\csname PY@tok@nt\endcsname{\let\PY@bf=\textbf\def\PY@tc##1{\textcolor[rgb]{0.00,0.50,0.00}{##1}}}
\expandafter\def\csname PY@tok@nd\endcsname{\def\PY@tc##1{\textcolor[rgb]{0.67,0.13,1.00}{##1}}}
\expandafter\def\csname PY@tok@s\endcsname{\def\PY@tc##1{\textcolor[rgb]{0.73,0.13,0.13}{##1}}}
\expandafter\def\csname PY@tok@sd\endcsname{\let\PY@it=\textit\def\PY@tc##1{\textcolor[rgb]{0.73,0.13,0.13}{##1}}}
\expandafter\def\csname PY@tok@si\endcsname{\let\PY@bf=\textbf\def\PY@tc##1{\textcolor[rgb]{0.73,0.40,0.53}{##1}}}
\expandafter\def\csname PY@tok@se\endcsname{\let\PY@bf=\textbf\def\PY@tc##1{\textcolor[rgb]{0.73,0.40,0.13}{##1}}}
\expandafter\def\csname PY@tok@sr\endcsname{\def\PY@tc##1{\textcolor[rgb]{0.73,0.40,0.53}{##1}}}
\expandafter\def\csname PY@tok@ss\endcsname{\def\PY@tc##1{\textcolor[rgb]{0.10,0.09,0.49}{##1}}}
\expandafter\def\csname PY@tok@sx\endcsname{\def\PY@tc##1{\textcolor[rgb]{0.00,0.50,0.00}{##1}}}
\expandafter\def\csname PY@tok@m\endcsname{\def\PY@tc##1{\textcolor[rgb]{0.40,0.40,0.40}{##1}}}
\expandafter\def\csname PY@tok@gh\endcsname{\let\PY@bf=\textbf\def\PY@tc##1{\textcolor[rgb]{0.00,0.00,0.50}{##1}}}
\expandafter\def\csname PY@tok@gu\endcsname{\let\PY@bf=\textbf\def\PY@tc##1{\textcolor[rgb]{0.50,0.00,0.50}{##1}}}
\expandafter\def\csname PY@tok@gd\endcsname{\def\PY@tc##1{\textcolor[rgb]{0.63,0.00,0.00}{##1}}}
\expandafter\def\csname PY@tok@gi\endcsname{\def\PY@tc##1{\textcolor[rgb]{0.00,0.63,0.00}{##1}}}
\expandafter\def\csname PY@tok@gr\endcsname{\def\PY@tc##1{\textcolor[rgb]{1.00,0.00,0.00}{##1}}}
\expandafter\def\csname PY@tok@ge\endcsname{\let\PY@it=\textit}
\expandafter\def\csname PY@tok@gs\endcsname{\let\PY@bf=\textbf}
\expandafter\def\csname PY@tok@gp\endcsname{\let\PY@bf=\textbf\def\PY@tc##1{\textcolor[rgb]{0.00,0.00,0.50}{##1}}}
\expandafter\def\csname PY@tok@go\endcsname{\def\PY@tc##1{\textcolor[rgb]{0.53,0.53,0.53}{##1}}}
\expandafter\def\csname PY@tok@gt\endcsname{\def\PY@tc##1{\textcolor[rgb]{0.00,0.27,0.87}{##1}}}
\expandafter\def\csname PY@tok@err\endcsname{\def\PY@bc##1{\setlength{\fboxsep}{0pt}\fcolorbox[rgb]{1.00,0.00,0.00}{1,1,1}{\strut ##1}}}
\expandafter\def\csname PY@tok@kc\endcsname{\let\PY@bf=\textbf\def\PY@tc##1{\textcolor[rgb]{0.00,0.50,0.00}{##1}}}
\expandafter\def\csname PY@tok@kd\endcsname{\let\PY@bf=\textbf\def\PY@tc##1{\textcolor[rgb]{0.00,0.50,0.00}{##1}}}
\expandafter\def\csname PY@tok@kn\endcsname{\let\PY@bf=\textbf\def\PY@tc##1{\textcolor[rgb]{0.00,0.50,0.00}{##1}}}
\expandafter\def\csname PY@tok@kr\endcsname{\let\PY@bf=\textbf\def\PY@tc##1{\textcolor[rgb]{0.00,0.50,0.00}{##1}}}
\expandafter\def\csname PY@tok@bp\endcsname{\def\PY@tc##1{\textcolor[rgb]{0.00,0.50,0.00}{##1}}}
\expandafter\def\csname PY@tok@fm\endcsname{\def\PY@tc##1{\textcolor[rgb]{0.00,0.00,1.00}{##1}}}
\expandafter\def\csname PY@tok@vc\endcsname{\def\PY@tc##1{\textcolor[rgb]{0.10,0.09,0.49}{##1}}}
\expandafter\def\csname PY@tok@vg\endcsname{\def\PY@tc##1{\textcolor[rgb]{0.10,0.09,0.49}{##1}}}
\expandafter\def\csname PY@tok@vi\endcsname{\def\PY@tc##1{\textcolor[rgb]{0.10,0.09,0.49}{##1}}}
\expandafter\def\csname PY@tok@vm\endcsname{\def\PY@tc##1{\textcolor[rgb]{0.10,0.09,0.49}{##1}}}
\expandafter\def\csname PY@tok@sa\endcsname{\def\PY@tc##1{\textcolor[rgb]{0.73,0.13,0.13}{##1}}}
\expandafter\def\csname PY@tok@sb\endcsname{\def\PY@tc##1{\textcolor[rgb]{0.73,0.13,0.13}{##1}}}
\expandafter\def\csname PY@tok@sc\endcsname{\def\PY@tc##1{\textcolor[rgb]{0.73,0.13,0.13}{##1}}}
\expandafter\def\csname PY@tok@dl\endcsname{\def\PY@tc##1{\textcolor[rgb]{0.73,0.13,0.13}{##1}}}
\expandafter\def\csname PY@tok@s2\endcsname{\def\PY@tc##1{\textcolor[rgb]{0.73,0.13,0.13}{##1}}}
\expandafter\def\csname PY@tok@sh\endcsname{\def\PY@tc##1{\textcolor[rgb]{0.73,0.13,0.13}{##1}}}
\expandafter\def\csname PY@tok@s1\endcsname{\def\PY@tc##1{\textcolor[rgb]{0.73,0.13,0.13}{##1}}}
\expandafter\def\csname PY@tok@mb\endcsname{\def\PY@tc##1{\textcolor[rgb]{0.40,0.40,0.40}{##1}}}
\expandafter\def\csname PY@tok@mf\endcsname{\def\PY@tc##1{\textcolor[rgb]{0.40,0.40,0.40}{##1}}}
\expandafter\def\csname PY@tok@mh\endcsname{\def\PY@tc##1{\textcolor[rgb]{0.40,0.40,0.40}{##1}}}
\expandafter\def\csname PY@tok@mi\endcsname{\def\PY@tc##1{\textcolor[rgb]{0.40,0.40,0.40}{##1}}}
\expandafter\def\csname PY@tok@il\endcsname{\def\PY@tc##1{\textcolor[rgb]{0.40,0.40,0.40}{##1}}}
\expandafter\def\csname PY@tok@mo\endcsname{\def\PY@tc##1{\textcolor[rgb]{0.40,0.40,0.40}{##1}}}
\expandafter\def\csname PY@tok@ch\endcsname{\let\PY@it=\textit\def\PY@tc##1{\textcolor[rgb]{0.25,0.50,0.50}{##1}}}
\expandafter\def\csname PY@tok@cm\endcsname{\let\PY@it=\textit\def\PY@tc##1{\textcolor[rgb]{0.25,0.50,0.50}{##1}}}
\expandafter\def\csname PY@tok@cpf\endcsname{\let\PY@it=\textit\def\PY@tc##1{\textcolor[rgb]{0.25,0.50,0.50}{##1}}}
\expandafter\def\csname PY@tok@c1\endcsname{\let\PY@it=\textit\def\PY@tc##1{\textcolor[rgb]{0.25,0.50,0.50}{##1}}}
\expandafter\def\csname PY@tok@cs\endcsname{\let\PY@it=\textit\def\PY@tc##1{\textcolor[rgb]{0.25,0.50,0.50}{##1}}}

\def\PYZbs{\char`\\}
\def\PYZus{\char`\_}
\def\PYZob{\char`\{}
\def\PYZcb{\char`\}}
\def\PYZca{\char`\^}
\def\PYZam{\char`\&}
\def\PYZlt{\char`\<}
\def\PYZgt{\char`\>}
\def\PYZsh{\char`\#}
\def\PYZpc{\char`\%}
\def\PYZdl{\char`\$}
\def\PYZhy{\char`\-}
\def\PYZsq{\char`\'}
\def\PYZdq{\char`\"}
\def\PYZti{\char`\~}
% for compatibility with earlier versions
\def\PYZat{@}
\def\PYZlb{[}
\def\PYZrb{]}
\makeatother


    % Exact colors from NB
    \definecolor{incolor}{rgb}{0.0, 0.0, 0.5}
    \definecolor{outcolor}{rgb}{0.545, 0.0, 0.0}



    
    % Prevent overflowing lines due to hard-to-break entities
    \sloppy 
    % Setup hyperref package
    \hypersetup{
      breaklinks=true,  % so long urls are correctly broken across lines
      colorlinks=true,
      urlcolor=urlcolor,
      linkcolor=linkcolor,
      citecolor=citecolor,
      }
    % Slightly bigger margins than the latex defaults
    
    \geometry{verbose,tmargin=1in,bmargin=1in,lmargin=1in,rmargin=1in}
    
    

    \begin{document}
    
    
    \maketitle
    
    

    
    \hypertarget{modsim-project-2-optimal-configuration-of-a-second-order-high-pass-filter}{%
\section{ModSim Project 2: Optimal Configuration of a Second Order High
Pass
Filter}\label{modsim-project-2-optimal-configuration-of-a-second-order-high-pass-filter}}

\hypertarget{shreya-chowdhary-and-kyle-mccracken}{%
\subsection{Shreya Chowdhary and Kyle
McCracken}\label{shreya-chowdhary-and-kyle-mccracken}}

    \hypertarget{what-is-the-configuration-of-resistors-and-capacitors-in-a-second-order-high-pass-filter-that-would-result-in-a-bode-plot-that-aligns-perfectly-or-as-close-as-possible-to-the-theoretical-model-for-a-high-pass-filter}{%
\subsubsection{What is the configuration of resistors and capacitors in
a second order high pass filter that would result in a bode plot that
aligns perfectly (or as close as possible) to the theoretical model for
a high pass
filter?}\label{what-is-the-configuration-of-resistors-and-capacitors-in-a-second-order-high-pass-filter-that-would-result-in-a-bode-plot-that-aligns-perfectly-or-as-close-as-possible-to-the-theoretical-model-for-a-high-pass-filter}}

The bode plots for second order high pass filters don't exactly match
the theoretical model for a second-order high pass filter (while
first-order high pass filters do match the theoretical model). This is
due to a exchange of current between the two high-pass filters. Thus, we
decided to create a model that would determine the optimal combination
of resistors and capacitors for a given tau to minimize current flow
between the two high pass filters in order to minimize the error between
the theoretical bode plot and the simulated bode plot.

    \hypertarget{methodology}{%
\subsubsection{Methodology}\label{methodology}}

We consider the following circuit:

In this circuit, we use \(V_{in}\) where \(V_{in} = A\sin{2\pi{ft}}\) to
represent the sinusoidal input waveform, \(V_m\) to represent the
voltage at the branch between the two high-pass filters, and \(V_{out}\)
to represent the output voltage. We use our voltages as the stocks for
our stock and flow diagram (shown below).

    Through circuit analysis, we derived differential equations for the
flows:

\[\frac{dV_{in}}{dt} = 2\pi{Af\cos{2\pi{ft}}}\]
\[\frac{dV_m}{dt} = \frac{dV_{in}}{dt} - \frac{\frac{V_m}{R_1} + \frac{V_{out}}{R_2}}{C_1}\]
\[\frac{dV_{out}}{dt} = \frac{dV_{out}}{dt} - \frac{V_{out}}{R_2C_2}\]

The only assumption we made was a certain amount of instantaneousness in
the change of voltage, which is a valid assumption based on our
knowledge of how voltage responds to current.

    \begin{Verbatim}[commandchars=\\\{\}]
{\color{incolor}In [{\color{incolor} }]:} \PY{c+c1}{\PYZsh{} Configure Jupyter so figures appear in the notebook}
        \PY{o}{\PYZpc{}}\PY{k}{matplotlib} inline
        
        \PY{c+c1}{\PYZsh{} Configure Jupyter to display the assigned value after an assignment}
        \PY{o}{\PYZpc{}}\PY{k}{config} InteractiveShell.ast\PYZus{}node\PYZus{}interactivity=\PYZsq{}last\PYZus{}expr\PYZus{}or\PYZus{}assign\PYZsq{}
        
        \PY{k+kn}{from} \PY{n+nn}{modsim} \PY{k}{import} \PY{o}{*}
        
        \PY{c+c1}{\PYZsh{} Creating the initial state}
        \PY{n}{init} \PY{o}{=} \PY{n}{State}\PY{p}{(}\PY{n}{Vm}\PY{o}{=}\PY{l+m+mi}{0}\PY{p}{,}
                     \PY{n}{Vout}\PY{o}{=}\PY{l+m+mi}{0}\PY{p}{)}
        
        \PY{c+c1}{\PYZsh{} Initializing the configuration of the high pass filter}
        \PY{n}{params} \PY{o}{=} \PY{n}{Params}\PY{p}{(}\PY{n}{R1}\PY{o}{=}\PY{l+m+mi}{1000000}\PY{p}{,}
                        \PY{n}{R2}\PY{o}{=}\PY{l+m+mi}{1000000}\PY{p}{,}
                        \PY{n}{C1}\PY{o}{=}\PY{l+m+mf}{1e\PYZhy{}11}\PY{p}{,}
                        \PY{n}{C2}\PY{o}{=}\PY{l+m+mf}{1e\PYZhy{}11}\PY{p}{)}
        
        
        \PY{c+c1}{\PYZsh{} Initializing the system configurations }
        \PY{c+c1}{\PYZsh{} freqs is the number of frequencies to sweep through }
        \PY{c+c1}{\PYZsh{} numwavels is the number of wavelengths simulated per frequency }
        \PY{c+c1}{\PYZsh{} stepres is ...?}
        \PY{n}{setsystem} \PY{o}{=} \PY{n}{System}\PY{p}{(}\PY{n}{tau} \PY{o}{=} \PY{l+m+mf}{1.5e\PYZhy{}7}\PY{p}{,}
                           \PY{n}{A}\PY{o}{=}\PY{l+m+mf}{0.5}\PY{p}{,} \PY{c+c1}{\PYZsh{}Amplitude of Vin wave}
                           \PY{n}{init}\PY{o}{=}\PY{n}{init}\PY{p}{,}
                           \PY{n}{t0}\PY{o}{=}\PY{l+m+mi}{0}\PY{p}{,}
                           \PY{n}{freqs} \PY{o}{=} \PY{l+m+mi}{8}\PY{p}{,}
                           \PY{n}{stepres} \PY{o}{=} \PY{l+m+mi}{200}\PY{p}{,}
                           \PY{n}{numwavels} \PY{o}{=} \PY{l+m+mi}{4}\PY{p}{)}
        
        \PY{k}{def} \PY{n+nf}{make\PYZus{}system}\PY{p}{(}\PY{n}{params}\PY{p}{,} \PY{n}{setsystem}\PY{p}{)}\PY{p}{:}
            \PY{l+s+sd}{\PYZsq{}\PYZsq{}\PYZsq{}}
        \PY{l+s+sd}{    Creates a system object.}
        \PY{l+s+sd}{    \PYZsq{}\PYZsq{}\PYZsq{}}
            \PY{c+c1}{\PYZsh{} param params: the configuration for the circuit (R1, R2)}
                
            \PY{c+c1}{\PYZsh{} Creates a system object representing the circuit with the correct configuration.}
            \PY{n+nb}{print}\PY{p}{(}\PY{l+s+s1}{\PYZsq{}}\PY{l+s+s1}{making system}\PY{l+s+s1}{\PYZsq{}}\PY{p}{)}
            \PY{n}{R1}\PY{p}{,} \PY{n}{R2}\PY{p}{,} \PY{n}{C1}\PY{p}{,} \PY{n}{C2} \PY{o}{=} \PY{n}{params}
        
            \PY{c+c1}{\PYZsh{} Sets the configuration of the filter }
            \PY{n}{setsystem}\PY{o}{.}\PY{n}{set}\PY{p}{(}\PY{n}{params} \PY{o}{=} \PY{n}{params}\PY{p}{)}
            
            \PY{c+c1}{\PYZsh{} Determines the characteristic cut off frequency }
            \PY{n}{fc} \PY{o}{=} \PY{l+m+mi}{1}\PY{o}{/}\PY{p}{(}\PY{l+m+mi}{2}\PY{o}{*}\PY{n}{np}\PY{o}{.}\PY{n}{pi}\PY{o}{*}\PY{n}{setsystem}\PY{o}{.}\PY{n}{tau}\PY{p}{)}
            
            \PY{c+c1}{\PYZsh{} Determines the maximum and minimum for the range of frequencies }
            \PY{c+c1}{\PYZsh{} to sweep when creating the bode plot}
            \PY{c+c1}{\PYZsh{} These frequencies wil be 2 orders of magnitude above and }
            \PY{c+c1}{\PYZsh{} below the cut\PYZhy{}off frequency}
            \PY{n}{flow} \PY{o}{=} \PY{n+nb}{int}\PY{p}{(}\PY{n}{np}\PY{o}{.}\PY{n}{log10}\PY{p}{(}\PY{n}{fc}\PY{p}{)}\PY{p}{)}\PY{o}{\PYZhy{}}\PY{l+m+mi}{2}
            \PY{n}{fhigh} \PY{o}{=} \PY{n+nb}{int}\PY{p}{(}\PY{n}{np}\PY{o}{.}\PY{n}{log10}\PY{p}{(}\PY{n}{fc}\PY{p}{)}\PY{p}{)}\PY{o}{+}\PY{l+m+mi}{2}
            
            \PY{n}{setsystem}\PY{o}{.}\PY{n}{set}\PY{p}{(}\PY{n}{f1} \PY{o}{=} \PY{n}{flow}\PY{p}{,} \PY{n}{f2} \PY{o}{=} \PY{n}{fhigh}\PY{p}{)}
            \PY{n+nb}{print}\PY{p}{(}\PY{l+s+s1}{\PYZsq{}}\PY{l+s+s1}{made system}\PY{l+s+s1}{\PYZsq{}}\PY{p}{)}
            \PY{n}{system} \PY{o}{=} \PY{n}{setsystem}
        
            \PY{k}{return} \PY{n}{system}
        
        \PY{k}{def} \PY{n+nf}{slope\PYZus{}func}\PY{p}{(}\PY{n}{init}\PY{p}{,} \PY{n}{t}\PY{p}{,} \PY{n}{system}\PY{p}{)}\PY{p}{:}
            \PY{l+s+sd}{\PYZsq{}\PYZsq{}\PYZsq{}}
        \PY{l+s+sd}{    Determines the ODEs for Vin, Vm, and Vout, which will be used }
        \PY{l+s+sd}{    later for the ODE solver.}
        \PY{l+s+sd}{    \PYZsq{}\PYZsq{}\PYZsq{}}
            \PY{c+c1}{\PYZsh{} param init: the initial state of the system}
            \PY{c+c1}{\PYZsh{} param t: the time at which this function will be evaluated}
            \PY{c+c1}{\PYZsh{} param system: the system object}
        
            \PY{n}{unpack}\PY{p}{(}\PY{n}{system}\PY{p}{)}
            \PY{n}{R1}\PY{p}{,} \PY{n}{R2}\PY{p}{,} \PY{n}{C1}\PY{p}{,} \PY{n}{C2} \PY{o}{=} \PY{n}{system}\PY{o}{.}\PY{n}{params}
        
            \PY{n}{vm}\PY{p}{,} \PY{n}{vout} \PY{o}{=} \PY{n}{init}
        
            \PY{c+c1}{\PYZsh{} ODEs for Vin, Vm, Vout}
            \PY{c+c1}{\PYZsh{} We treat dVin as a variable because Vin is not influenced by Vm or Vout, but influences both}
            \PY{n}{dvin} \PY{o}{=} \PY{l+m+mi}{2} \PY{o}{*} \PY{n}{np}\PY{o}{.}\PY{n}{pi} \PY{o}{*} \PY{n}{A} \PY{o}{*} \PY{n}{f} \PY{o}{*} \PY{n}{np}\PY{o}{.}\PY{n}{cos}\PY{p}{(}\PY{l+m+mi}{2}\PY{o}{*}\PY{n}{np}\PY{o}{.}\PY{n}{pi}\PY{o}{*}\PY{n}{f}\PY{o}{*}\PY{n}{t}\PY{p}{)}
            \PY{n}{dvm} \PY{o}{=} \PY{n}{dvin} \PY{o}{\PYZhy{}} \PY{p}{(}\PY{n}{vm}\PY{o}{/}\PY{n}{R1} \PY{o}{+} \PY{n}{vout}\PY{o}{/}\PY{n}{R2}\PY{p}{)} \PY{o}{/} \PY{n}{C1}
            \PY{n}{dvout} \PY{o}{=} \PY{n}{dvm} \PY{o}{\PYZhy{}} \PY{n}{vout} \PY{o}{/} \PY{p}{(}\PY{n}{R2}\PY{o}{*}\PY{n}{C2}\PY{p}{)}
        
            \PY{k}{return} \PY{n}{dvm}\PY{p}{,} \PY{n}{dvout}
        
        \PY{k}{def} \PY{n+nf}{run\PYZus{}bode}\PY{p}{(}\PY{n}{system}\PY{p}{)}\PY{p}{:}
            \PY{l+s+sd}{\PYZsq{}\PYZsq{}\PYZsq{}}
        \PY{l+s+sd}{    Generates a bode plot for the given set of frequencies.}
        \PY{l+s+sd}{    \PYZsq{}\PYZsq{}\PYZsq{}}
            \PY{c+c1}{\PYZsh{} param system: the system object}
                
            \PY{n}{unpack}\PY{p}{(}\PY{n}{system}\PY{p}{)}
            \PY{n+nb}{print}\PY{p}{(}\PY{l+s+s1}{\PYZsq{}}\PY{l+s+s1}{start bode}\PY{l+s+s1}{\PYZsq{}}\PY{p}{)}
            
            \PY{c+c1}{\PYZsh{} Creates a set of frequencies on a log scale}
            \PY{n}{farray} \PY{o}{=} \PY{n}{np}\PY{o}{.}\PY{n}{logspace}\PY{p}{(}\PY{n}{f1}\PY{p}{,} \PY{n}{f2}\PY{p}{,} \PY{n}{freqs}\PY{p}{)}
        
            \PY{n}{Re} \PY{o}{=} \PY{n}{TimeSeries}\PY{p}{(}\PY{p}{)}
        
            \PY{k}{for} \PY{n}{f} \PY{o+ow}{in} \PY{n}{farray}\PY{p}{:}
                \PY{n}{system}\PY{o}{.}\PY{n}{set}\PY{p}{(}\PY{n}{f}\PY{o}{=}\PY{n}{f}\PY{p}{,} \PY{n}{t\PYZus{}end} \PY{o}{=} \PY{n}{numwavels} \PY{o}{/} \PY{n}{f}\PY{p}{)}
        
                \PY{c+c1}{\PYZsh{} Determines the maximum step length, to force the bode plot to run faster}
                \PY{n}{max\PYZus{}step} \PY{o}{=} \PY{p}{(}\PY{n}{system}\PY{o}{.}\PY{n}{t\PYZus{}end} \PY{o}{\PYZhy{}} \PY{n}{t0}\PY{p}{)} \PY{o}{/} \PY{p}{(}\PY{n}{stepres}\PY{p}{)}
                \PY{n}{results}\PY{p}{,} \PY{n}{details} \PY{o}{=} \PY{n}{run\PYZus{}ode\PYZus{}solver}\PY{p}{(}\PY{n}{system}\PY{p}{,} \PY{n}{slope\PYZus{}func}\PY{p}{,} \PY{n}{max\PYZus{}step} \PY{o}{=} \PY{n}{max\PYZus{}step}\PY{p}{)}
        
                \PY{c+c1}{\PYZsh{} Uses nfev for the \PYZsh{} of steps and to select out the tail}
                \PY{n}{tail} \PY{o}{=} \PY{n+nb}{int}\PY{p}{(}\PY{n}{details}\PY{o}{.}\PY{n}{nfev}\PY{o}{/}\PY{p}{(}\PY{l+m+mi}{2}\PY{o}{*}\PY{n}{np}\PY{o}{.}\PY{n}{pi}\PY{o}{*}\PY{n}{numwavels}\PY{p}{)}\PY{p}{)}
                \PY{n}{amplitudeM} \PY{o}{=} \PY{n}{results}\PY{o}{.}\PY{n}{Vout}\PY{o}{.}\PY{n}{tail}\PY{p}{(}\PY{n}{tail}\PY{p}{)}\PY{o}{.}\PY{n}{ptp}\PY{p}{(}\PY{p}{)}
                \PY{n}{Re}\PY{p}{[}\PY{n}{f}\PY{p}{]} \PY{o}{=} \PY{n}{amplitudeM}
                
            \PY{n+nb}{print}\PY{p}{(}\PY{l+s+s1}{\PYZsq{}}\PY{l+s+s1}{done bode}\PY{l+s+s1}{\PYZsq{}}\PY{p}{)}
            \PY{k}{return} \PY{n}{Re}
        
        \PY{k}{def} \PY{n+nf}{run\PYZus{}freq}\PY{p}{(}\PY{n}{system}\PY{p}{)}\PY{p}{:}
            \PY{l+s+sd}{\PYZsq{}\PYZsq{}\PYZsq{}}
        \PY{l+s+sd}{    Runs the simulation for a specific frequency (in this case, the cut off frequency).}
        \PY{l+s+sd}{    \PYZsq{}\PYZsq{}\PYZsq{}}
            
            \PY{n+nb}{print}\PY{p}{(}\PY{l+s+s1}{\PYZsq{}}\PY{l+s+s1}{start freq}\PY{l+s+s1}{\PYZsq{}}\PY{p}{)}
            
            \PY{c+c1}{\PYZsh{} Calculates the cut off frequency }
            \PY{n}{fc} \PY{o}{=} \PY{l+m+mi}{1}\PY{o}{/}\PY{p}{(}\PY{l+m+mi}{2}\PY{o}{*}\PY{n}{np}\PY{o}{.}\PY{n}{pi} \PY{o}{*} \PY{n}{system}\PY{o}{.}\PY{n}{tau}\PY{p}{)}
            \PY{n+nb}{print}\PY{p}{(}\PY{n}{fc}\PY{p}{)}
            
            \PY{c+c1}{\PYZsh{} Sets the frequency to run the bode plot at to the cut off}
            \PY{n}{system}\PY{o}{.}\PY{n}{set}\PY{p}{(}\PY{n}{f} \PY{o}{=} \PY{n}{fc}\PY{p}{)}
            
            \PY{n}{system}\PY{o}{.}\PY{n}{set}\PY{p}{(}\PY{n}{t\PYZus{}end} \PY{o}{=} \PY{n}{system}\PY{o}{.}\PY{n}{numwavels} \PY{o}{/} \PY{n}{system}\PY{o}{.}\PY{n}{f}\PY{p}{)}
            \PY{n}{unpack}\PY{p}{(}\PY{n}{system}\PY{p}{)}
        
        
            \PY{n}{max\PYZus{}step} \PY{o}{=} \PY{p}{(}\PY{n}{t\PYZus{}end} \PY{o}{\PYZhy{}} \PY{n}{t0}\PY{p}{)} \PY{o}{/} \PY{p}{(}\PY{n}{stepres}\PY{p}{)}
        
            \PY{n}{results}\PY{p}{,} \PY{n}{details} \PY{o}{=} \PY{n}{run\PYZus{}ode\PYZus{}solver}\PY{p}{(}\PY{n}{system}\PY{p}{,} \PY{n}{slope\PYZus{}func}\PY{p}{,} \PY{n}{max\PYZus{}step} \PY{o}{=} \PY{n}{max\PYZus{}step}\PY{p}{)}
            \PY{n}{tail} \PY{o}{=} \PY{n+nb}{int}\PY{p}{(}\PY{n}{details}\PY{o}{.}\PY{n}{nfev}\PY{o}{/}\PY{p}{(}\PY{l+m+mi}{2}\PY{o}{*}\PY{n}{np}\PY{o}{.}\PY{n}{pi}\PY{o}{*}\PY{n}{numwavels}\PY{p}{)}\PY{p}{)}
            \PY{n}{amplitudeM} \PY{o}{=} \PY{n}{results}\PY{o}{.}\PY{n}{Vout}\PY{o}{.}\PY{n}{tail}\PY{p}{(}\PY{n}{tail}\PY{p}{)}\PY{o}{.}\PY{n}{ptp}\PY{p}{(}\PY{p}{)}
        
            \PY{n+nb}{print}\PY{p}{(}\PY{l+s+s1}{\PYZsq{}}\PY{l+s+s1}{done freq}\PY{l+s+s1}{\PYZsq{}}\PY{p}{)}
            \PY{k}{return} \PY{n}{amplitudeM}
        
        
        \PY{k}{def} \PY{n+nf}{run\PYZus{}calc}\PY{p}{(}\PY{n}{system}\PY{p}{)}\PY{p}{:}
            \PY{l+s+sd}{\PYZsq{}\PYZsq{}\PYZsq{}}
        \PY{l+s+sd}{    Determines what the theoretical model should look like for the given frequencies}
        \PY{l+s+sd}{    \PYZsq{}\PYZsq{}\PYZsq{}}
            \PY{n}{unpack}\PY{p}{(}\PY{n}{system}\PY{p}{)}
            \PY{n}{farray} \PY{o}{=} \PY{n}{np}\PY{o}{.}\PY{n}{logspace}\PY{p}{(}\PY{n}{f1}\PY{p}{,} \PY{n}{f2}\PY{p}{,} \PY{n}{freqs}\PY{p}{)}
            \PY{n}{C} \PY{o}{=} \PY{n}{TimeSeries}\PY{p}{(}\PY{p}{)}
        
            \PY{k}{for} \PY{n}{f} \PY{o+ow}{in} \PY{n}{farray}\PY{p}{:}
                
                \PY{c+c1}{\PYZsh{} Uses circuit theory to calculate the amplitude}
                \PY{n}{w} \PY{o}{=} \PY{l+m+mi}{2}\PY{o}{*}\PY{n}{np}\PY{o}{.}\PY{n}{pi}\PY{o}{*}\PY{n}{f}
                \PY{n}{rcw1} \PY{o}{=} \PY{n}{tau}\PY{o}{*}\PY{n}{w}
                \PY{n}{rcw2} \PY{o}{=} \PY{n}{tau}\PY{o}{*}\PY{n}{w}
                \PY{n}{amplitudeC} \PY{o}{=} \PY{p}{(}\PY{n}{rcw1}\PY{o}{*}\PY{n}{rcw2}\PY{p}{)} \PY{o}{/} \PY{p}{(}\PY{n}{np}\PY{o}{.}\PY{n}{sqrt}\PY{p}{(}\PY{l+m+mi}{1} \PY{o}{+} \PY{n}{rcw1}\PY{o}{*}\PY{o}{*}\PY{l+m+mi}{2}\PY{p}{)} \PY{o}{*} \PY{n}{np}\PY{o}{.}\PY{n}{sqrt}\PY{p}{(}\PY{l+m+mi}{1} \PY{o}{+} \PY{n}{rcw2}\PY{o}{*}\PY{o}{*}\PY{l+m+mi}{2}\PY{p}{)}\PY{p}{)}
                \PY{n}{C}\PY{p}{[}\PY{n}{f}\PY{p}{]} \PY{o}{=} \PY{n}{amplitudeC}
        
            \PY{k}{return} \PY{n}{C}
        
        \PY{k}{def} \PY{n+nf}{error\PYZus{}func}\PY{p}{(}\PY{n}{params}\PY{p}{,} \PY{n}{setsystem}\PY{p}{)}\PY{p}{:}
            \PY{l+s+sd}{\PYZsq{}\PYZsq{}\PYZsq{}}
        \PY{l+s+sd}{    Determines the error between the simulated bode plot and the theoretical bode plot.}
        \PY{l+s+sd}{    \PYZsq{}\PYZsq{}\PYZsq{}}
            \PY{n}{system} \PY{o}{=} \PY{n}{make\PYZus{}system}\PY{p}{(}\PY{n}{params}\PY{p}{,} \PY{n}{setsystem}\PY{p}{)}
            \PY{n+nb}{print}\PY{p}{(}\PY{n}{params}\PY{p}{)}
            \PY{n+nb}{print}\PY{p}{(}\PY{l+s+s1}{\PYZsq{}}\PY{l+s+s1}{start error}\PY{l+s+s1}{\PYZsq{}}\PY{p}{)}
            \PY{c+c1}{\PYZsh{}ampM = run\PYZus{}freq(system)}
            \PY{c+c1}{\PYZsh{}ampC = system.A}
        
            \PY{n}{results} \PY{o}{=} \PY{n}{run\PYZus{}bode}\PY{p}{(}\PY{n}{setsystem}\PY{p}{)}
            \PY{n}{calcs} \PY{o}{=} \PY{n}{run\PYZus{}calc}\PY{p}{(}\PY{n}{setsystem}\PY{p}{)}
        
            \PY{n}{errors} \PY{o}{=} \PY{n}{results} \PY{o}{\PYZhy{}} \PY{n}{calcs}
            \PY{n+nb}{print}\PY{p}{(}\PY{l+s+s1}{\PYZsq{}}\PY{l+s+s1}{done error}\PY{l+s+s1}{\PYZsq{}}\PY{p}{)}
            \PY{k}{return} \PY{n}{errors}
        
        \PY{k}{def} \PY{n+nf}{plot\PYZus{}bode\PYZus{}plot}\PY{p}{(}\PY{p}{)}\PY{p}{:}
            \PY{l+s+sd}{\PYZsq{}\PYZsq{}\PYZsq{}}
        \PY{l+s+sd}{    Plots the simulated bode plot against the theoretical model}
        \PY{l+s+sd}{    \PYZsq{}\PYZsq{}\PYZsq{}}
            \PY{n}{fig} \PY{o}{=} \PY{n}{plt}\PY{o}{.}\PY{n}{figure}\PY{p}{(}\PY{p}{)}
            \PY{n}{ax} \PY{o}{=} \PY{n}{fig}\PY{o}{.}\PY{n}{add\PYZus{}subplot}\PY{p}{(}\PY{l+m+mi}{111}\PY{p}{)}
            \PY{n}{ax}\PY{o}{.}\PY{n}{set\PYZus{}xscale}\PY{p}{(}\PY{l+s+s1}{\PYZsq{}}\PY{l+s+s1}{log}\PY{l+s+s1}{\PYZsq{}}\PY{p}{)}
            \PY{n}{lns1} \PY{o}{=} \PY{n}{ax}\PY{o}{.}\PY{n}{plot}\PY{p}{(}\PY{n}{results}\PY{p}{,} \PY{n}{label} \PY{o}{=} \PY{l+s+s1}{\PYZsq{}}\PY{l+s+s1}{simulated}\PY{l+s+s1}{\PYZsq{}}\PY{p}{)}
            \PY{n}{lns2} \PY{o}{=} \PY{n}{ax}\PY{o}{.}\PY{n}{plot}\PY{p}{(}\PY{n}{run\PYZus{}calc}\PY{p}{(}\PY{n}{system}\PY{p}{)}\PY{p}{,} \PY{n}{label} \PY{o}{=} \PY{l+s+s1}{\PYZsq{}}\PY{l+s+s1}{calculation}\PY{l+s+s1}{\PYZsq{}}\PY{p}{)}
            \PY{n}{lns} \PY{o}{=} \PY{n}{lns1}\PY{o}{+}\PY{n}{lns2}
            \PY{n}{labs} \PY{o}{=} \PY{p}{[}\PY{n}{l}\PY{o}{.}\PY{n}{get\PYZus{}label}\PY{p}{(}\PY{p}{)} \PY{k}{for} \PY{n}{l} \PY{o+ow}{in} \PY{n}{lns}\PY{p}{]}
            \PY{n}{ax}\PY{o}{.}\PY{n}{legend}\PY{p}{(}\PY{n}{lns}\PY{p}{,} \PY{n}{labs}\PY{p}{,} \PY{n}{loc} \PY{o}{=} \PY{l+s+s1}{\PYZsq{}}\PY{l+s+s1}{best}\PY{l+s+s1}{\PYZsq{}}\PY{p}{)}
            \PY{n}{savefig}\PY{p}{(}\PY{l+s+s1}{\PYZsq{}}\PY{l+s+s1}{graphs}\PY{l+s+s1}{\PYZbs{}}\PY{l+s+s1}{BodePlotLowRes2.png}\PY{l+s+s1}{\PYZsq{}}\PY{p}{)}
        
        \PY{n}{best\PYZus{}params}\PY{p}{,} \PY{n}{fit\PYZus{}details} \PY{o}{=} \PY{n}{fit\PYZus{}leastsq}\PY{p}{(}\PY{n}{error\PYZus{}func}\PY{p}{,} \PY{n}{params}\PY{p}{,} \PY{n}{setsystem}\PY{p}{,} \PY{n}{maxfev} \PY{o}{=} \PY{l+m+mi}{500}\PY{p}{)}
        \PY{n+nb}{print}\PY{p}{(}\PY{n}{best\PYZus{}params}\PY{p}{,} \PY{n}{setsystem}\PY{o}{.}\PY{n}{tau}\PY{o}{/}\PY{n}{best\PYZus{}params}\PY{o}{.}\PY{n}{R1}\PY{p}{,} \PY{n}{setsystem}\PY{o}{.}\PY{n}{tau}\PY{o}{/}\PY{n}{best\PYZus{}params}\PY{o}{.}\PY{n}{R2}\PY{p}{)}
        
        \PY{n}{system} \PY{o}{=} \PY{n}{make\PYZus{}system}\PY{p}{(}\PY{n}{best\PYZus{}params}\PY{p}{,} \PY{n}{setsystem}\PY{p}{)}
        \PY{n}{results} \PY{o}{=} \PY{n}{run\PYZus{}bode}\PY{p}{(}\PY{n}{system}\PY{p}{)}
        \PY{n}{plot\PYZus{}bode\PYZus{}plot}\PY{p}{(}\PY{p}{)}
\end{Verbatim}


    \hypertarget{results}{%
\subsubsection{Results}\label{results}}

According to our model, the optimal configuration for resistors and
capacitors in a second order high pass filter with a tau of 1.5e-7 would
be an \(R_1\) and \(R_2\) of 1e4 and a \(C_1\) and \(C_2\) of 1.5e-11.

    \hypertarget{interpretation}{%
\subsubsection{Interpretation}\label{interpretation}}

One issue we encountered was with the model choosing Rs and Cs that were
negative. We circumvented this by forcing the C-values to be positive,
thus preventing the differential equations from working out correctly if
the R-values are negative.

The model ends up basically eliminating the second filter entirely, so
in future iterations, we would put more constraints on the Rs and Cs it
could choose to prevent it from choosing Rs and Cs for the second filter
that are effectively close to 0. This is due to the fact that we were
unable to force the selected R's and C's to match the tau that we set.

In future iterations of the model, we might try to increase the realism
of the model. For example, there is a standard set of capacitor values
used in electrical engineering, so we might try to force the model to
choose exclusively these capacitors and vary the resistor values more
freely (rather than the other way around).

    \hypertarget{abstract}{%
\subsubsection{Abstract}\label{abstract}}

What is the configuration of resistors and capacitors in a second order
high pass filter that would result in a bode plot that aligns perfectly
(or as close as possible) to the theoretical model for a high pass
filter?

Our model settles upon resistors that are large enough to reduce the
current enough for the bode to match. It also chooses a larger second
resistor to further reduce the effect that current has on the bode plot.

For a high pass filter with a tau of 1.5e-5, our model finds values as
follows: * R1 = 1.5e5 * R2 = 1.474e7 * C1 = 1.00e-10 * C2 = 1.02e-12

\includegraphics{attachment:BodePlotLowRes4\%20-\%20Copy.png} Amplitude
vs Frequency for the modelled circuitry.

The results that come out of our model make intuitive sense; however, it
is very hard to determine whether the model has truly optimized the
circuit or if it has just found resistor and capacitor values that work
well. Either way our model is functionalin that it will produce results
that are valuable for designing and building a real second-order high
pass filter.


    % Add a bibliography block to the postdoc
    
    
    
    \end{document}
